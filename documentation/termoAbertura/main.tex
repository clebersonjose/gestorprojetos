\documentclass{article}
\usepackage{geometry}

\geometry {
    paper=a4paper,
    inner=0.5cm,
    outer=0.8cm,
    bindingoffset=.5cm,
    top=3cm,
    bottom=2cm,
    left=2cm,
    right=2cm,
    marginpar=0.8cm,
}

\title {
  Termo de Abertura do Projeto \\
  \large
    Sistema de Gerenciamento de Projetos
}

\author {
  CLÉBERSON JOSÉ DE ANDRADE \\
  \large
    clebersonjosea@gmail.com
}

\date{19 de Outubro de 2020}


\begin{document}

\pagenumbering{gobble}
  \maketitle
  \newpage
\pagenumbering{arabic}

\section{Descrição do Projeto em Alto Nível}
  \paragraph{}
    A empresa da Juliana esta sentindo dificuldade para decidir a ordem de suas atividades nos projetos em desenvolvimento da empresa, ela juntamente com sua equipe decidiu contratar uma equipe de TI para elaborar em software que facilite essa decisão e por consequência diminua o tempo médio da geração de valor por meio da execução das atividades na ordem mais adequada aos projetos.

\section{Propósito ou Justificativa do Projeto}
  \paragraph{}
    O projeto tem com propósito desenvolver um sistema de gerenciamento de projetos, que permita monitorar e acompanhar tarefas, bem como ordena-las usando Matriz de Eisenhower e Matriz Esforço x Impacto, para garantir que as prioridades estão sendo seguidas e gerar valor em menos tempo.

\section{Objetivos do Projeto}
  \paragraph{}
    O objetivo do projeto é diminuir o tempo de produção dos MVP's dos projetos, bem como a criação de novas funcionalidades e melhorias, sendo esperado uma redução na quantidades de tarefas que não criam valor direto aos MVP's e novos projetos.

\section{Benefícios}
  \paragraph{}
    Tarefas sequenciadas pelas suas prioridades e impactos no negocio, e geração de valor em menos espaço de tempo.

\section{Requisitos de alto nível}
  \begin{enumerate}
    \item O sistema deve seguir o modelo Kanban, com colunas de estado com as tarefas;

    \item O sistema deve permitir a criação de colunas, as colunas devem ter nome e ordem de exibição;

    \item O sistema deve permitir a criação de tarefas, as tarefas devem ter nome, coluna de estado, data de entrega, nível de prioridade (de 0 à 5), nível de impacto no projeto (de 0 à 5), pontos de esforço (cada ponto de esforço equivale a 1h) e responsavel pela tarefa;

    \item O sistema deve ordenar as tarefas usando Matriz de Eisenhower e Matriz Esforço x Impacto;

    \item O sistema deve permitir o cadastro de novos usuários;

    \item O sistema deve funcionar em sistemas Linux, Windows, MacOS, Android e IOS;

    \item O sistema deve estar sempre disponível.
  \end{enumerate}

\section{Premissas iniciais}
  \begin{enumerate}
    \item  O gerente deverá levantar requisitos com o patrocínador;

    \item  A gestão do projeto será responsabilidade do Noah Nicolas das Neves;
  \end{enumerate}

\section{Restrições iniciais}
  \begin{enumerate}
    \item Todas as atividades do projeto deverão ser executadas obedecendo as boas práticas de desenvolvimento de software.
  \end{enumerate}

\section{Limites do Projeto}
  \paragraph{}
    O sistema deve ser uma ferramenta de gerenciamento de atividades de projetos, usando Kanban, tendo como entregas-chave o quadro Kanban.

\section{Riscos de Alto Nível}
  \begin{enumerate}
    \item Atraso do desenvolvimento;
    \item Falta de equipe qualificada.
  \end{enumerate}

\section{Resumo do Cronograma de Marcos}
  \begin{enumerate}
    \item Desenvolvimento da documentação do projeto;
    \item Desenvolvimento do banco de dados;
    \item Desenvolvimento do back-end do software;
    \item Desenvolvimento da interface web;
    \item Desenvolvimento da documentação do software.
  \end{enumerate}

\section{Resumo do Orçamento}
  \paragraph{}
    Está previsto um orçamento de R\$ 20.000,00 para o projeto.

\section{Lista das Partes Interessadas}
  \begin{enumerate}
    \item Noah Nicolas das Neves;
    \item Juliana Andreia Esther Ferreira;
    \item Yasmin Fátima Barros;
    \item Guilherme João Eduardo Nogueira;
    \item Yago João Gomes.
  \end{enumerate}

\section{Requisitos para Aprovação do Projeto}
  \begin{enumerate}
    \item Todos os requisitos funcionais e não funcionais devem ser atendidas.;
    \item Aprovação do gerente do projeto;
    \item Aprovação do patrocinador do projeto.
  \end{enumerate}

\section{Gerente do Projeto}
  \begin{enumerate}
    \item Noah Nicolas das Neves (Gerente do Projeto)
  \end{enumerate}

\section{Patrocinador}
  \begin{enumerate}
    \item Juliana Andreia Esther Ferreira (Autoridade Máxima, pode modificar o projeto)
  \end{enumerate}

%\approvalsignature{Juliana Andreia Esther Ferreira}{Patrocinador / Gerente Projeto}
\section{Autorização}
\paragraph{}
  Aprovado por: \\ \\ \\ \\
  \begin{tabular}{@{}p{0in}p{3.5in}p{0in}p{2in}@{}}
    &\hrulefill  &  &\hrulefill \\
    &Juliana Andreia Esther Ferreira  &  &Data \\
    &Patrocinador / Gerente Projeto & &
  \end{tabular}

\end{document}
